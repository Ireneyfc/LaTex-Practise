
\documentclass[12pt, a4paper]{article}
\usepackage{fontspec, xunicode, xltxtra}  
\usepackage{xeCJK}%中文字体
\usepackage{fontspec}
\usepackage[slantfont, boldfont]{xeCJK}
\usepackage{CJKfntef}


% 设置英文字体
\setmainfont{Times New Roman}   %西文默认衬线字体(serif)
\setsansfont{Arial}   %西文默认无衬线字体(sans serif)
\setmonofont{Courier New}           %西文默认的等宽字体
%分别对应\rmfamliy,\sffamily,\ttfamliy

% 设置中文字体
\setCJKmainfont[ItalicFont={楷体}, BoldFont={黑体}]{宋体}
\setCJKsansfont{黑体}
\setCJKmonofont{仿宋_GB2312}%中文等宽字体

%-----------------------xeCJK下设置中文字体------------------------------%
\setCJKfamilyfont{song}{SimSun}                             %宋体 song  
\newcommand{\song}{\CJKfamily{song}}                        
\setCJKfamilyfont{fs}{FangSong_GB2312}                      %仿宋2312 fs  
\newcommand{\fs}{\CJKfamily{fs}}                            
\setCJKfamilyfont{yh}{Microsoft YaHei UI}                    %微软雅黑 yh  
\newcommand{\yh}{\CJKfamily{yh}}  
\setCJKfamilyfont{hei}{SimHei}                              %黑体  hei  
\newcommand{\hei}{\CJKfamily{hei}}    
\setCJKfamilyfont{hwxh}{STXihei}                                %华文细黑  hwxh  
\newcommand{\hwxh}{\CJKfamily{hwxh}}
\setCJKfamilyfont{asong}{Adobe Song Std}                        %Adobe 宋体  asong  
\newcommand{\asong}{\CJKfamily{asong}}
\setCJKfamilyfont{ahei}{Adobe Heiti Std}                            %Adobe 黑体  ahei  
\newcommand{\ahei}{\CJKfamily{ahei}}  
\setCJKfamilyfont{akai}{Adobe Kaiti Std}                            %Adobe 楷体  akai  
\newcommand{\akai}{\CJKfamily{akai}}
\newfontfamily\ls{LiSu}                                            % 2.隶书 ls


%------------------------------设置字体大小------------------------%  
\newcommand{\chuhao}{\fontsize{42pt}{\baselineskip}\selectfont}     %初号  
\newcommand{\xiaochuhao}{\fontsize{36pt}{\baselineskip}\selectfont} %小初号  
\newcommand{\yihao}{\fontsize{28pt}{\baselineskip}\selectfont}      %一号  
\newcommand{\erhao}{\fontsize{21pt}{\baselineskip}\selectfont}      %二号  
\newcommand{\xiaoerhao}{\fontsize{18pt}{\baselineskip}\selectfont}  %小二号  
\newcommand{\sanhao}{\fontsize{15.75pt}{\baselineskip}\selectfont}  %三号  
\newcommand{\sihao}{\fontsize{14pt}{\baselineskip}\selectfont}       %四号  
\newcommand{\xiaosihao}{\fontsize{12pt}{\baselineskip}\selectfont}  %小四号  
\newcommand{\wuhao}{\fontsize{10.5pt}{\baselineskip}\selectfont}    %五号  
\newcommand{\xiaowuhao}{\fontsize{9pt}{\baselineskip}\selectfont}   %小五号  
\newcommand{\liuhao}{\fontsize{7.875pt}{\baselineskip}\selectfont}  %六号  
\newcommand{\qihao}{\fontsize{5.25pt}{\baselineskip}\selectfont}    %七号

% 中文断行设置
\XeTeXlinebreaklocale "zh"
\XeTeXlinebreakskip = 0pt plus 1pt

\title{测试}
\author{尹方晨}
\date{2016年6月6日}
\begin{document}
\maketitle
\begin{center}
满纸荒唐言\\
一把辛酸泪\\
都云作者痴\\
谁解其中味\\ 
\end{center}
\begin{verse}
\texttt{Courier New\quad Stray birds of summer come to my window to sing and fly away}. \\
\textsf{Arial\quad And yellow leaves of autumn, which have no songs}, \\
\textrm{Times New Roman\quad flutter and fall there with a sign}.\\
\hfill \emph{RabindranathTagore}
\end{verse}
\begin{verse}
\texttt{夏天的飞鸟},\textsf{飞到我的窗前唱歌},\textrm{又飞去了}。\\
秋天的黄叶,它们没有什么可唱,只叹息一声,飞落在那里。\\
\hfill \emph{罗宾德拉纳特·泰戈尔}%\emph 命令表示强调,用于把直立体改为意大利体
\end{verse}


\underline{下划线强调}\\
%对应的分别是下划线,双下划线,波浪线,划去线,着重点
\uline{Fashion}\\
\uuline{Fashion}\\
\uwave{Fashion 黑}\\
\sout{Fashion}\\
\dotuline{Fashion }\\
%注意:一般情况下中文英文的强调命令差别不大,但是着重点有区别
\dotuline{大佬}\\
\CJKunderdot{大佬}\\
%常用的几个命令有\Huge,\huge,\LARGE,\large,\normalsize,\small,\tiny
{\tiny Fashion}\quad {\normalsize Fashion} \quad {\huge Fashion}\\
%如果我要修改的再细一点,那么我们可以选择命令\fontsize{}{},第一个里面是字体的大小(默认单位pt,10pt是正常的默认字体大小),第二个里面是行距。
%注意:字体大小与行距息息相关。对于英文文档类(如article)来说,行距默认是字体大小的1.2倍,对于中文来说则是1.56倍。

罗宾德拉纳特·泰戈尔\\
\song{song:罗宾德拉纳特·泰戈尔}\\
\fs{fs:罗宾德拉纳特·泰戈尔}\\
\yh{yh:罗宾德拉纳特·泰戈尔}\\
\hei{hei:罗宾德拉纳特·泰戈尔}\\
\hwxh{hwxh:罗宾德拉纳特·泰戈尔}\\
\asong{asong:罗宾德拉纳特·泰戈尔}\\
\ahei{ahei:罗宾德拉纳特·泰戈尔}\\
\akai{akai:罗宾德拉纳特·泰戈尔}\\
\akai{akai:罗宾德拉纳特·泰戈尔} \quad \ls{ls:罗宾德拉纳特·泰戈尔}\\

\end{document}