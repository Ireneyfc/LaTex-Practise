% https://github.com/mhyee/latex-examples
% https://github.com/davidstutz/latex-resources学习资源
\documentclass[UTF8,a6paper]{ctexart}
\usepackage{graphicx}
\usepackage{float}
\usepackage{amsmath}
\usepackage{cite}
\usepackage{geometry}
\geometry{a6paper,centering,scale=0.8}
\usepackage{caption}
\usepackage[nottoc]{tocbibind}
\newtheorem{thm}{定理}
\title{\heiti 杂谈勾股定理}
\author{\kaishu 张三}
\date{\today}
\bibliographystyle{plain}

\begin{document}
\maketitle
\begin{abstract}
这是一篇关于勾股定理的小短文。
\end{abstract}
\tableofcontents
\section{勾股定理在古代}
\label{sec:ancient}
西方称勾股定理为毕达哥拉斯定理,将勾股定理的发现归功于公元前6世纪的毕达哥拉斯学派\cite{Kline}。该学派得到一个法则,可以求出可以排成直角三角形三边的三元数组。毕达哥拉斯学派没有书面著作,该定理的严格表述和证明则见于欧几里德\footnote{欧几里德,约公元前330-275年}《几何原本》的命题47:“直角三角形斜边上的正方形等于两只脚边上的两个正方形之和。”正面是用面积做的。\par
我国《周髀算经》载商高(约公元前12世纪)答周公问:
\begin{quote}
\zihao{-5}\kaishu 勾广三,股修四,径隅五。
\end{quote}
又载陈子(约公元前7-6世纪)答荣方问:
\begin{quote}
\zihao{-5}\kaishu 若求邪至日者,以日下为勾,日高为股,勾股各自乘,并而开方除之,得邪至日。
\end{quote}
都较古希腊更早。后者已经明确道出勾股定理的一般形式。图\ref{fig:gougudingli}是我国古代对勾股定理的一种证明\cite{quanjing}。
\begin{figure}[ht]
\centering
\includegraphics[width=3cm]{gougudingli.pdf}
\caption{\small\it 宋赵爽在《周髀算经》注中作的弦图(仿制),该图给出了勾股定理的极具对称美的证明。}
\label{fig:gougudingli}
\end{figure}
\section{勾股定理的现代形式}
勾股定理可以用现代语言表述如下:
\begin{thm}[勾股定理]
直角三角形斜边的平方等于两腰的平方和。\par
可以用符号语言表述为:设直角三角形ABC,其中\angle C=$90^\circ$,则有
\begin{equation}\label{eq:gougu}
AB^2=BC^2+AC^2
\end{equation}
\end{thm}
满足式\eqref{eq:gougu}的整数称为\emph{勾股数}。第\ref{sec:ancient}节所说毕达哥拉斯学派得到的三元数组就是勾股数。下表列出了一些较小的勾股数:
\begin{table}[H]
\begin{tabular}{|rrr|}
\hline
直角边$a$  &  直角边$b$  &  直角边$c$\\
\hline
3&4&5\\
5&12&13\\\hline
\end{tabular}
\qquad
($a^2+b^2=c^2$)
\end{table}
\nocite{Shiye}
\bibliography{math}
\end{document}
——————————————————分割线——————————————————————
math.bib文件
@BOOK{Kline,
 title={古今数学思想},
 publisher={上海科学技术出版社},
 year={2002},
 author={克莱因}
 }

 @ARTICLE{quanjing,
  author={曲安京},
  title={商高、赵爽与刘徽关于勾股定理的证明},
  journal={数学传播},
  year={1998},
  volume={20},
  number={3}
  }

@BOOK{Shiye,
 title={几何的有名定理},
 publisher={上海科学技术出版社},
 year={1986},
 author={矢野健太郎}
 }




 