\documentclass{article}
\usepackage{ctex} % 必须使用ctex 宏包
\usepackage{fontspec}

% 下面的设置将使用指定的字体族设置文章中对应需要的部分。例如英文正文部分将全部采用`Times New Roman`字体
\setmainfont{Times New Roman}     
%\setsansfont{Dotum}               
%\setmonofont{Droid Sans Fallback} 

\setCJKsansfont{Adobe 楷体 Std}
\setCJKmonofont{LiHei Pro}
% 设置中文主体部分的字体为Adobe宋体,但是对于字形斜体和粗体不使用Adobe宋体的默认字体而是采用方正姚体和方正舒体
\setCJKmainfont[BoldFont={方正舒体}, ItalicFont={方正姚体}, BoldItalicFont={Adobe Heiti Std}]{Adobe 宋体 Std}
% \setCJKmainfont[BoldFont={LiHei Pro Bold}, ItalicFont={LiHei Pro Italic}, BoldItalicFont={LiHer Pro Bold Italic}]{LiHei Pro}


\setCJKfamilyfont{hwhp}{华文琥珀}
\newcommand{\hwhp}{\CJKfamily{hwhp}}

%有时候我们的论文会在不同的电脑之间来回编辑,不同电脑的字体不一定完全包含有tex文件需要的字体,这时候为每一台电脑安装指定的字体将是一个非常繁杂的工作。你可以将你需要的字体放在你的论文目录下,然后通过字体文件使用字体。这样还能在版本管理工具git上方便使用。
%你如在你的论文中存在fonts目录用来存放字体SourceSansPro-SemiboldIt.otf,你可以使用
%\newfontfamily\semib[Path=fonts/]{SourceSansPro-SemiboldIt.otf}

\begin{document}

下面的文章内容关于的是$\LaTeXe$文章中的字体,你可以自定义字体。setmainfont主要定义英文字体。这里的默认字体为Times New Roman。
中文字体主要为LiHei Pro。下面的字体的不同形状展示,\textbf{粗体},\textit{斜体}。下面的内容主要是展示粗体部分文字:
\textbf{这里写一个故事:从前有座山,山里有座庙,庙里有个老和尚和小和尚讲故事。}看到了吧,即使主题字体设置的是Adobe宋体但是粗体的时候是按照方正舒体设置的。里面的BoldFont之类的字体均可以改成你想用的字体。下面斜体版本的故事:
\textit{从前有座山,山里有座庙,庙里有个老和尚和小和尚讲故事。}实际上每个字体都会有响应的粗体、斜体、倾斜之类的字体。在Word里面我们可以直接设置字形。但是在LaTeX中如果你不加设置,粗体的字体将是你指定的字体的粗体。例如上面的例子中如果不设置,主体字体的粗体将显示LiHei Pro的粗体。

如果你想指定任意字体,你可以自定义CJKfamily。例如:
{\hwhp 从前有座山,山里有座庙,庙里有个老和尚和小和尚讲故事。}

\end{document}
